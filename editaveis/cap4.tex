\chapter*[DATA WAREHOUSE COMO APOIO À ATIVIDADES DE MEDIÇÃO]{DATA WAREHOUSE COMO APOIO À ATIVIDADES DE MEDIÇÃO}
\addcontentsline{toc}{chapter}{DATA WAREHOUSE COMO APOIO À ATIVIDADES DE MEDIÇÃO}
Quanto mais os dados de métricas são compartilhados entre os projetos da organização, mais abrangente e eficaz é a capacidade da organização para avaliar os seus projetos. Neste contexto, RUIZ, SILVEIRA, & BECKER (2010) afirmam que é crucial coletar e armazenar dados relacionados a diferentes projetos de acordo com a visão unificada da organização. E referenciam KIMBALL & ROSS (2002) afirmando que a solução para o maior grau de integração e de normalização é representado por abordagens de Data Warehousing, onde as métricas são estruturadas de acordo com um modelo multidimensional.
Um Data Warehouse (DW) é uma coleção de dados históricos integrados, não-voláteis, orientado a assunto, destinado a apoiar os processos de tomada de decisão durante um período específico de tempo. (KIMBALL & ROSS, 2002)

\begin{itemize}
	\item Integrado: Múltiplas fontes de dados que devem ser limpas, tratadas, convertidas, formatadas, e resumidas antes de serem armazenadas. (KIMBALL & ROSS, 2002)
	\item Orientado a assunto: organiza os dados pelos principais elementos do negócio para que os dados sejam analisados filtrando dados relevantes e excluindo os não úteis para a tomada de decisão. (KIMBALL & ROSS, 2002)
	\item Não-volátil: No DW, os dados são carregados e acessados, mas as atualizações geralmente ocorrem no ambiente operacional. (KIMBALL & ROSS, 2002)
	\item Variante no tempo: possibilita a manutenção de uma perspectiva histórica dos dados (KIMBALL & ROSS, 2002).
\end{itemize}
Um repositório consolidado de dados é conhecido como DW (a Data Warehouse), e o termo DWing (data warehousing) é o processo para montagem e gerenciamento desse repositório. Esse processo complexo abrange, entre outras coisas: (a) modelagem de negócio, (b) extração, transformação e carregamento de dados (ETL), (c) ferramentas de análise voltadas para os usuários finais, e (d) gestão e manutenção do repositório. Resumindo: os dados provenientes de diversas fontes são coletados, selecionados, tratados, transformados, integrados, e por fim carregados no DW. ETL é um processo crítico e pode consumir até 85% de todo o esforço global na criação e manutenção de um ambiente DW (KIMBALL & ROSS, 2002).
Dois fatores que são de grande importância em um ambiente DW são os mecanismos OLAP (on-line analytical processing, em português processamento analítico on-line) que são valiosos para a análise de tendências passadas, pois transformam dados em informações eficientes, dinâmicas, flexíveis e úteis, ou seja, recuperam as informações que são exibidas por meio de interface para os interessados (RUIZ, SILVEIRA, & BECKER, 2010). O Conselho OLAP\footnote{O Conselho OLAP (OLAP Council) é uma organização sem fins lucrativos, patrocinada por vários fornecedores de ferramentas OLAP, cujo objetivo é promover a educação sobre a tecnologia OLAP.} define OLAP como 
[...] categoria da tecnologia de software que permite que analistas, gerentes e executivos obtenham, de maneira rápida, consistente e interativa, acesso a uma variedade de visualizações possíveis de informações que foi transformada de dados puros para refletir a dimensão real do empreendimento do ponto de vista do usuário.
E os mecanismos de monitoramento e previsão que são necessários para melhorar a percepção do desempenho atual, bem como para a previsibilidade e detecção precoce de um comportamento inesperado (RUIZ, SILVEIRA, & BECKER, 2010).
O DW pode ter um estrutura centralizada ou distribuída em camadas que são conhecidas como: (1) repositório; (2) apresentação, e (3) ETL. camada ETL é a responsável por capturar dados de várias fontes heterogeneas, transformas, integrar e carrega-los no DW consolidado. A camada de repositório representa o próprio DW, e a camada de apresentação fornece funcionalidades analiticas para acessar o DW.
No modelo centralizado de um DW, o poder de processamento é maior e os processos de busca de informação podem ser otimizados, como pode ser visto na figura 11.A arquitetura em camadas é mais flexível e permite consultas simultâneas sem muita perda de performance. Na primeira camada é disponibilizado o servidor que atende a maior parte das consultas, com baixo volume de dados. Nas demais camadas tem-se os servidores com volume maior de dados que atenderão a uma quantidade menor de usuários, como pode ser visto na figura 12.
BARBALHO (2003) ainda cita que a construção de um DW passa por quatro fases principais:
\begin{itemize}
	\item Levantamento – avalia, junto aos tomadores de decisão, os conhecimentos que desejam ser adquiridos.
	\item Modelagem Multidimensional – representa-se a ideia central e suas dimensões onde identifica-se as questões principais e define-se como os dados serão armazenados.
	\item ETL (Extract, Transform and Load) – extração dos dados nos sistemas corporativos e transformação para carga no Data Warehouse.
	\item Visualização do Resultado – ferramentas de interação com o usuário através de interfaces amigáveis são disponibilizadas.
\end{itemize}

Um DW é constituido segundo uma estrutura dimensional que é representada pelos componentes principais: fatos e dimenções, onde uma tabela de dimensões armazena descrições textuais do negócio e as de fato as medidas númericas do negócio. (BARBALHO, 2003)
Segundo KIMBALL & ROSS (2002), a tabela de fatos é a principal tabela de um modelo dimensional, onde as medições numéricas de interesse da organização estão armazenadas, ou seja, a tabela é usada para representar uma medição de negócio. A tabela de fatos registra os fatos que serão analisados. É composta por uma chave primária e pelas métricas de interesse para o negócio. Em uma tabela de fatos, uma linha corresponde a uma medição. Uma medição é uma linha em um tabela de fatos. Todas as medições em uma tabela de fatos devem estar alinhadas na mesma granularidade.
KIMBALL & ROSS (2002) afirmam que a tabela de dimensão contém as descrições textuais do negócio, seus atributos são fonte das restrições das consultas, agrupamento dos resultados, e cabeçalhos para relatórios, e dizem ainda que cada dimensão deve ser identificada com uma única chave primária. 
Segundo KIMBALL (2002) existem três esquemas para modelagem multidimensional:
\begin{itemize}
	\item Modelo Estrela: uma tabela fato circundada de um conjunto de tabelas dimensão. O modelo estrela é caracterizado por uma ou mais tabelas fato e uma série de tabelas dimensão menores. (KIMBALL & ROSS, 2002) (OLIVEIRA, 2007)
	\item Modelo Floco de Neve: modelo simplificado do modelo estrela, onde a hierarquia dimensional é normalizada em um conjunto de tabelas dimensão menores, formando uma forma semelhante a um floco de neve. (KIMBALL & ROSS, 2002) (OLIVEIRA, 2007)
	\item Modelo Constelação de Fatos: várias tabelas fato compartilham tabelas dimensão, visto como uma coleção de estrelas, por isso chamado de constelação de fatos. (KIMBALL & ROSS, 2002) (OLIVEIRA, 2007)
\end{itemize}

A Erro: Origem da referência não encontrada ilustra um modelo estrela que se aplicado em modelo multidimensional de dados, através de uma das visualizações OLAP (drill down, drill up, slice, dice), é possível criar uma visão no formato de cubo, conhecida como Cubo, onde um exemplo pode ser visualizado na Erro: Origem da referência não encontrada. 







Um ambiente DW contribui para a representação de métricas, sejam elas de produto, processo ou projeto, em um modelo multidimensional, contribui também para diminuir a intromissão nos processos de ETL, no apoio às atividades de monitoramento que possuem uma frequente e grande quantidade de carga de dados, consistência e confiabilidade dos dados, entre outros (RUIZ, SILVEIRA, & BECKER, 2010).
Um escopo de medição, no qual as métricas de projeto, produto ou processo serão coletadas deve envolver quatro perspectivas: (1) o time de desenvolvimento; (2) o projeto; (3) a organização; (4) e as inter-organizações (RUIZ, SILVEIRA, & BECKER, 2010). Esse Trabalho de conclusão de curso explorou as métricas de custo, escopo e tempo que fazem parte da perspectiva de projeto.
A intrusão pode ser interpretada como um nível de envolvimento de recursos e humanos durante a execução do processo de captura de dados e RUIZ, SILVEIRA, & BECKER  (2010) referenciam (GOPAL, MUKHOPADHYAY, & KRISHNAN, 2005) que afirmam que intrusão é  um dos fatores cruciais de aceitação de um plano de métricas.
A idéia de diminuir a intrusão está associada ao fato de se utilizar o evento da Reunião Diária do SCRUM, onde o SCRUMMaster coletará a informação de quantidade horas restantes estimada para conclusão das tarefas em andamento no fluxo do Kanban, com cada integrante do time. Este seria o único momento de coleta que dependeria da participação do time.















