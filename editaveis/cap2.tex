\chapter*[GERENCIAMENTO DE PROJETO DE SOFTWARE]{GERENCIAMENTO DE PROJETO DE SOFTWARE}
\addcontentsline{toc}{chapter}{GERENCIAMENTO DE PROJETO DE SOFTWARE}

Perguntas tais como: 1) Quanto tempo levará para desenvolver o software? 2) Quanto custará o desenvolvimento do software? dentre outras, são algumas das perguntas as quais todos patrocinadores querem saber, pois antes de comprometer os recursos destinados a um projeto de desenvolvimento ou manutenção de software, o patrocinador deseja ter uma estimativa (PFLEEGER, 2004).
Já Tom DeMarco (2009), argumenta que ao invés de perguntarmos “Como controlar um projeto de software?” a pergunta mais importante a ser feita é “Por que estamos fazendo projetos que custam tão pouco?
Um projeto é um esforço temporário empreendido para criar um produto, serviço ou resultado exclusivo. Ou seja, um projeto é qualquer atividade com início, meio e fim e cujo resultado deve ser único. Temporário, porém, não significa necessariamente de curta duração, e nem se aplica ao produto, serviço ou resultado criado pelo projeto. (PMI, 2008)
A crescente aceitação do gerenciamento de projetos indica que a aplicação de conhecimento, processos, habilidades, ferramentas, e técnicas adequados pode ter um impacto significativo no sucesso de um projeto. (PMI, 2008) A área de gerenciamento de projetos é uma das áreas de conhecimento que mais cresce em utilização no mundo, sendo, hoje, objetivo de investimento em capacitação e metodologia pela maioria das empresas. Em todo projeto, é senso comum que uma das principais dificuldades está na medição e na avaliação dos resultados obtidos, sejam eles resultados finais ou parciais (durante sua execução) nos prazos, custos, qualidade, escopo, risco e outros. (VARGAS R. V., 2011)
De acordo com o Guia Project Management Body of Knowledge-PMBoK (PMI, 2008), gerenciamento de projeto é a aplicação de conhecimento, habilidades, ferramentas e técnicas às atividades do projeto a fim de atender aos seus requisitos. Gerenciar um projeto inclui: identificar requisitos; adaptar às diferentes necessidades à medida que o projeto é planejado e realizado; balancear as restrições conflitantes do projeto que incluem, mas não se limitam a: escopo, qualidade, tempo, custo.
VARGAS (2009), descreve sucintamente que a proposta do gerenciamento de projetos é estabelecer um processo estruturado e lógico para lidar com eventos que se caracterizam pela novidade, complexidade e dinâmica ambiental. Um dos fatores que impulsionam o gerenciamento de projetos é o crescimento da competitividade: quem for mais rápido e competente certamente conseguirá os melhores resultados. Essa competividade incita as empresas a conseguirem resultados com menos recursos, tempo e cada vez mais qualidade, ou seja, fazer mais que os concorrentes, gastando menos.
Para se aplicar os conceitos de controle de processo faz-se necessário diferenciar duas modalidades, não concomitantes, aplicáveis ao gerenciamento de projetos: metodologia com processo definidos (ou prescritivos) e a metodologia empírica. De acordo com MARTINS (2007), a primeira define o contexto do projeto e o escopo das entradas logo no início do projeto, enquanto que a outra define as entregas de forma abrangente e superficial, começando por um contexto inicial, que evolui e se adapta ao longo da execução.

\section{GERENCIAMENTO TRADICIONAL}
A metodologia com processos definidos e prescritivos, também chamada de abordagem tradicional, é a mais adequada em situações onde os passos a serem executados, em geral, são conhecidos, como por exemplo, na implantação de uma infraestrutura de TI. Instalação de um sistema como ERP ou CRM. Em projetos tradicionais, certo conjunto de entradas produzirá um conjunto específico de saídas. (MARTINS, 2007)

\subsection{O guia PMBOK}
O Guia PMBOK® fornece diretrizes para o gerenciamento de projetos individuais, define o gerenciamento e os conceitos relacionados e descreve o ciclo de vida do gerenciamento de projetos e os processos relacionados. Tem por objetivo ser um guia com um conjunto de conhecimentos e boas práticas de aplicação e não uma metodologia. É possível usar ferramentas e metodologias distintas para implementar sua estrutura. Como referência básica, a norma não é abrangente nem completa. Também fornece um vocabulário comum para se discutir, escrever e aplicar o gerenciamento de projetos entre os membros envolvidos. 
O guia é baseado em várias áreas e processos que organizam o trabalho a ser realizado durante o projeto. Os processos se relacionam e interagem segundo uma lógica definida para a condução do trabalho, realizada através de entradas, ferramentas e técnicas, e saídas.
Ciclo de vida e a organização do projeto 
O gerenciamento de projetos é feito num ambiente mais amplo que o projeto propriamente dito, seu ciclo de vida consiste nas fases que oferecem uma estrutura básica para o gerenciamento do projeto, independente do trabalho especifico envolvido, onde tamanho e complexidade não importam, pois todos os projetos podem ser mapeados para a estrutura do ciclo de vida: 1) Início do projeto; 2) Organização e preparação; 3) Execução do trabalho do projeto; e 4) Encerramento do projeto.  Esta estrutura pode ser visualizada na (PMI, 2008)Erro: Origem da referência não encontrada
Grupos de Processos ou Fases
Um processo é um conjunto de ações e atividades inter-relacionadas, que são executadas para alcançar um produto, resultado ou serviço predefinido. Cada processo é caracterizado por suas entradas, as ferramentas e as técnicas que podem ser aplicadas e as saídas resultantes.
Os processos de gerenciamento de projetos são agrupados em cinco categorias:
\begin{itemize}
	\item Iniciação: Processos realizados para definir um novo projeto ou uma nova fase de um projeto existente através da obtenção de autorização para iniciar o projeto ou a fase. Nos processos de iniciação, o escopo inicial é definido e os recursos financeiros iniciais são comprometidos;
	\item Planejamento: Processos realizados para definir o escopo do projeto, refinar os objetivos e desenvolver o curso de ação necessário para alcançar os objetivos para os quais o projeto foi criado. Os processos de planejamento desenvolvem o plano de gerenciamento e os documentos do projeto que serão usados para executá-lo. Composto pelas atividades: Desenvolver o plano de gerenciamento do projeto; Coletar os requisitos; Definir o escopo; Criar a estrutura analítica do projeto (EAP); Definir as atividades; Sequenciar as Atividades; Estimar os recursos das atividades; Estimar as durações das atividades; Desenvolver o cronograma; Estimar os custos; Determinar o orçamento; Planejar a qualidade; Desenvolver o plano de recursos humanos; Planejar as comunicações; Planejar o gerenciamento de riscos;  Identificar os riscos; Realizar a análise qualitativa de riscos; Realizar a análise quantitativa de riscos; Planejar respostas a riscos; Planejar as aquisições.;
	\item Execução: Processos realizados para executar o trabalho definido no plano de gerenciamento do projeto para satisfazer as especificações do mesmo;
	\item Monitoramento e Controle: Processos necessários para acompanhar, revisar e regular o progresso e o desempenho do projeto, identificar todas as áreas nas quais serão necessárias mudanças no plano e iniciar as mudanças correspondentes. O principal benefício deste grupo de processos é que o desempenho do projeto é observado e mensurado de forma periódica e uniforme para identificar variações em relação ao plano de gerenciamento do mesmo. Inclui também 1) Controlar as mudanças e recomendar ações preventivas em antecipação a possíveis problemas; 2) Monitorar as atividades do projeto em relação ao plano de gerenciamento e à linha de base de desempenho do mesmo; e, 3) Influenciar os fatores que poderiam impedir o controle integrado de mudanças, para que somente as mudanças aprovadas sejam implementadas;
	\item Encerramento: Os processos executados para finalizar todas as atividades de todos os grupos de processos, visando encerrar formalmente o projeto ou a fase.
\end{itemize}
Os processos de gerenciamento de projetos são apresentados como elementos distintos com interfaces bem definidas. Porém, na prática eles se sobrepõem e sua aplicação é iterativa e muitos deles são repetidos durante o projeto. A natureza integrativa do gerenciamento de projetos requer que o grupo de processos de monitoramento e controle interaja com os outros grupos de processos, o de iniciação começa o projeto e o de encerramento termina, e o de planejamento fornece ao grupo de processos de execução o plano de gerenciamento e os documentos do projeto à medida que o projeto avança, conforme mostra a (PMI, 2008)Erro: Origem da referência não encontrada.

\subsubsection{Áreas de Conhecimento}
As áreas de conhecimento na prática são interativas e podem se sobrepor e interagir. As nove áreas são mostradas na Erro: Origem da referência não encontrada.


Gerenciamento das Aquisições: Inclui os processos necessários para comprar ou adquirir produtos, serviços ou resultados externos à equipe do projeto. (PMI, 2008)
Gerenciamento das Comunicações: Inclui os processos necessários para assegurar que as informações do projeto sejam geradas, coletadas, distribuídas, armazenadas, recuperadas e organizadas de maneira oportuna e apropriada. (PMI, 2008)
Gerenciamento dos Custos: Inclui os processos envolvidos em estimativas, orçamentos e controle dos custos, de modo que o projeto possa ser terminado dentro do orçamento aprovado. (PMI, 2008)
Gerenciamento do Escopo: Inclui os processos necessários para assegurar que o projeto inclui todo o trabalho necessário, e apenas o necessário, para terminar o projeto com sucesso. (PMI, 2008)
Gerenciamento da Integração: Inclui os processos e as atividades necessárias para identificar, definir, combinar, unificar e coordenar os vários processos e atividades dos grupos de processos de gerenciamento. (PMI, 2008)
Gerenciamento da Qualidade do projeto: Inclui os processos e as atividades da organização executora que determinam as políticas de qualidade, os objetivos e as responsabilidades, de modo que o projeto satisfaça às necessidades para as quais foi empreendido. (PMI, 2008)
Gerenciamento dos Recursos Humanos: Inclui os processos que organizam e gerenciam a equipe do projeto. (PMI, 2008)
Gerenciamento dos Riscos: Inclui os processos de planejamento, identificação, análise, planejamento de respostas, monitoramento e controle de riscos de um projeto. (PMI, 2008)
Gerenciamento do Tempo: Inclui os processos necessários para gerenciar o término pontual do projeto. (PMI, 2008)

\section{GERENCIAMENTO ÁGIL}
Muitas empresas de desenvolvimento de software estão se esforçando para se tornar mais ágeis. Equipes ágeis bem sucedidas estão produzindo software de alta qualidade para melhor atender, mais rapidamente, as necessidades dos usuários e com um custo menor do que as equipes tradicionais. (COHN, 2010)
As empresas que tem tido sucesso na adoção de métodos ágeis, adotando um processo como o SCRUM combinado com outros métodos ágeis, estão evidenciando ganhos significativos de produtividade com diminuição de custos correspondentes. Elas são capazes de levar produtos ao mercado mais rapidamente e com um maior grau de satisfação do cliente, obtendo uma maior visibilidade para o processo de desenvolvimento, levando a uma maior previsibilidade, como consequência, e não como causa. E para estas empresas os termos “fora de controle”, “projetos que jamais serão concluídos” tornaram-se coisa do passado. (COHN, 2010)
Um processo empírico requer frequentemente inspeções e adaptações durante o projeto, que é definido de forma inexata e pode gerar resultados imprevisíveis e é mais adequado para projetos de inovação e criação de novos produtos, como o desenvolvimento de software, por exemplo. A metodologia empírica, um dos pilares do SCRUM, é indicada nas situações onde as entradas do processo variam e processo é muito complexo para produzir resultados semelhantes. (MARTINS, 2007)

\subsection{SCRUM}
SCRUM é um conjunto de princípios e práticas que auxiliam equipes a entregar produtos em ciclos curtos, favorecendo um feedback rápido, melhoria continua e rápida adaptação à mudanças. Como guia do método de gerenciamento ágil, o SCRUM tem sido predominantemente usado para desenvolvimento de software, mas também está provando ser eficaz em esforços muito além funcionando muito bem para qualquer escopo complexo e inovador de trabalho. (SCRUMALLIANCE, 2013)
O SCRUM se baseia na ideia de controle de processo empírico, isto é, utiliza o conceito de progresso por meio da capacidade produtiva do time, e não de predição, para planejar e agendar entregas. SCRUM é um framework que, ao invés de fornecer completas e detalhadas descrições de como tudo deverá ser feito no projeto, permite ao time decidir por si só como deverá ser feito o trabalho, possibilitando saber qual a melhor forma de como resolver o problema apresentado. (SCRUMMETHODOLOGY, 2013)
O SCRUM tem como alicerce a auto-organização e a multifuncionalidade da equipe: auto-organização, ou seja, a equipe é auto organizável, onde não há um líder geral que decide qual tarefa cada pessoa irá fazer, pois essas questões são decididas em equipe; multifuncionalidade, ou seja, cada membro da equipe precisa ter habilidade para pegar uma funcionalidade desde a ideia à implementação (MOUNTAINGOAT, 2013).O modelo de equipe no SCRUM é projetado para aperfeiçoar a flexibilidade, criatividade e produtividade. (SCHWABER & SYTHERLAND, 2011)
O SCRUM consiste em Equipes associadas a seus papéis, eventos, artefatos e regras. (SCHWABER & SYTHERLAND, 2011) E suporta todas as suas práticas em uma estrutura de modelo de ciclo de vida iterativo e incremental, como é mostrado na  (SCHWABER, 2004)Erro: Origem da referência não encontrada.
No SCRUM, no início da iteração, chamada de Sprint, o time revisa o que deve ser feito. Depois seleciona o que acreditam ser possível ser transformado em um incremento de uma funcionalidade potencialmente entregável ao final da iteração. O time é então deixado a sós para dar o seu melhor esforço até o término da sprint. Ao final da sprint o time apresenta o incremento da funcionalidade construída para que os stakeholders possam inspecionar a funcionalidade e que adaptações dentro do tempo possam ser efetuadas.
Durante o sprint o time observa os requisitos, considera a tecnologia disponível, e avalia suas próprias habilidades e capacidades. Posteriormente considera coletivamente como construir a funcionalidade, modificar diariamente, se necessário, sua abordagem ao ponto que se encontram novas complexidades, dificuldades e surpresas. O time descobre o que precisa ser feito e seleciona o melhor caminho para fazer. (SCHWABER, 2004)
O SCRUM possui três papeis: Product Owner, a Equipe de Desenvolvimento e o SCRUMMaster.
Product Owner
O Product Owner é responsável por representar os interesses de todo mundo que possua uma participação ou interesse, no projeto e no sistema resultante, ou seja, representa o negócio, clientes ou usuários, e orienta a equipe a construir o produto certo. (MOUNTAINGOAT, 2013) Ele fornece a inicial e contínua base do projeto apresentando os requisitos gerais do projeto, os objetivos do retorno sobre investimento (ROI) e os planos de entrega. O PO é o único responsável por gerenciar o Backlog do produto e deve garantir que a mais importante funcionalidade é produzida primeiro e construída. Isso é alcançado pela frequentemente priorização do Backlog do produto. (SCHWABER, 2004) 
O gerenciamento do Backlog do produto inclui: a) Expressar claramente os itens do Backlog do Produto; b) Ordenar os itens do Backlog do Produto para alcançar melhor as metas e missões; c) Garantir o valor do trabalho realizado pelo time de desenvolvimento; d) Garantir que o Backlog do Produto seja visível, transparente, claro para todos, e mostrar o que o Time vai trabalhar a seguir; e, e) Garantir que a Equipe de Desenvolvimento entenda os itens do Backlog do Produto no nível necessário. (SCHWABER & SYTHERLAND, 2011)
Time de Desenvolvimento
O time consiste de membros que realizam o trabalho, são auto gerenciáveis, auto organizáveis, indivisíveis em sub-equipes, multifuncionais e possuem a responsabilidade de tornar o Backlog do produto em um incremento funcional de software em um Sprint, além de gerenciar seus próprios trabalhos a serem feitos. O membros do time são responsáveis pelo sucesso de cada sprint e do projeto como um todo. (SCHWABER, 2004)
SCRUMMaster
O SCRUMMaster é responsável por garantir o uso dos princípios e práticas do SCRUM. É muitas vezes considerado um treinador para a equipe, ajudando-a a fazer o melhor trabalho que puder, isso envolve a remoção de quaisquer impedimentos ao progresso, facilitando reuniões e o trabalhar com o Product Owner para garantir que o Backlog do produto esteja em “boa forma” e pronto para a próxima Sprint. (MOUNTAINGOAT, 2013) Algumas características de um SCRUMMaster são: responsável; humilde; colaborativo; comprometido; influente; inteligente;  questionador; paciente; protetor; transparente. (COHN, 2010) (RUBIN, 2012) E algumas responsabilidades são: treinador; líder-servo; autoridade no processo; escudo de interferência; removedor de impedimentos; agente de mudanças. (RUBIN, 2012)
Eventos SCRUM
O SCRUM usa eventos de tempo fixo (time-boxed, em inglês), onde todo evento tem uma duração máxima. Isso garante que uma quantidade adequada de tempo seja gasta no planejamento sem permitir perda no processo de planejamento. Cada evento é uma oportunidade de inspecionar e adaptar alguma coisa, favorecendo uma transparência e inspeção criteriosa. Os eventos são: Sprint; Reunião de Planejamento da Sprint; Reunião Diária; Revisão da Sprint; Retrospectiva da Sprint.
Sprint
A Sprint tem o tempo fixo de 10 a 30 dias consecutivos. Além de outros fatores, esta é a quantidade de tempo necessária para uma equipe construir algo de interesse significativo para o Product Owner e trazê-lo para um estado onde ele é potencialmente utilizável. Este também é o tempo máximo que a maioria dos stakeholders vai esperar sem perder o interesse no progresso da equipe e sem perder a sua crença de que a equipe está fazendo algo significativo para eles. Durante a Sprint:
\begin{itemize}
	\item A equipe pode procurar aconselhamento fora, ajuda, informação e apoio; (SCHWABER, 2004)
	\item A equipe se compromete com o Backlog do Produto durante a reunião de planejamento da Sprint e ninguém está autorizado a altera-lo, pois o mesmo é congelado até o final da Sprint; (SCHWABER, 2004)
	\item Se a Sprint revela-se inviável, a SCRUMMaster pode decidir por finalizar a Sprint e iniciar uma nova reunião de planejamento da Sprint para iniciar a próxima Sprint. O SCRUMMaster pode fazer esta mudança de sua própria iniciativa ou quando solicitado pela equipe ou o proprietário do produto; (SCHWABER, 2004)
	\item Se a equipe sente-se incapaz de completar o comprometido com o Backlog do Produto durante a Sprint, pode-se consultar com o Product Owner sobre quais itens remover da Sprint atual, contrário também pode acontecer; (SCHWABER, 2004)
	\item Os membros da equipe têm duas responsabilidades administrativas durante o Sprint: 1) eles devem participar da Reunião Diária, e 2) manter o Backlog da Sprint atualizada e disponível, visível a todos. (SCHWABER, 2004)
	\item A composição da Equipe de Desenvolvimento permanecem constantes; (SCHWABER & SYTHERLAND, 2011)
	\item As metas de qualidade não diminuem. (SCHWABER & SYTHERLAND, 2011)
\end{itemize}

Reunião de Planejamento da Sprint
A reunião de planejamento do Sprint tem o tempo fixo de oito horas e é composta por dois segmentos que são de tempo fixo de quatro horas cada. Os participantes são os SCRUMMaster, o Product Owner e o Time. (SCHWABER & SYTHERLAND, 2011)
O primeiro segmento é para a seleção de Backlog do Produto, no qual a equipe deve selecionar os itens do Backlog do Produto que ela acredita que pode se comprometer para transforma-los em um incremento de funcionalidade do produto potencialmente entregável. (SCHWABER & SYTHERLAND, 2011)
As entradas da reunião de planejamento da Sprint são:
\begin{itemize}
	\item O Backlog do Produto, o qual o Product Owner deve preparar antes da reunião. Na ausência do Product Owner ou do Backlog do Produto, o SCRUMMaster é o encarregado de construir um Backlog do Produto adequado antes da reunião e substituir o Product Owner. A equipe pode fazer sugestões, mas a decisão do que é constituído o Backlog do Produto e o da Sprint é de responsabilidade do Product Owner. Já a equipe é responsável por determinar o quanto do Product Backlog a equipe tentará fazer durante o Sprint; (SCHWABER & SYTHERLAND, 2011) (SCHWABER, 2004)
	\item O mais recente incremento do produto, o qual a equipe vai demonstrar essa funcionalidade para o Product Owner na reunião de revisão da Sprint no final do Sprint; (SCHWABER & SYTHERLAND, 2011)
	\item A capacidade projetada pelo time durante a Sprint; e, (SCHWABER & SYTHERLAND, 2011)
	\item O desempenho anterior do time. (SCHWABER & SYTHERLAND, 2011)
\end{itemize}

O segundo segmento é para preparar um Backlog da Sprint. Ocorre imediatamente após o primeiro segmento e também tem tempo fixo de 4 horas. O Product Owner deve estar disponível para a equipe durante o segundo segmento para responder às perguntas que a equipe pode ter sobre o Product Backlog. Cabe à equipe, durante o segundo segmento em se comprometer em transformar o Product Backlog selecionado em um incremento de funcionalidade do produto potencialmente entregável. A saída do segundo segmento da reunião de planejamento do Sprint, segundo (SCHWABER, 2004) é:
\begin{itemize}
	\item Uma lista, chamada de Sprint Backlog;
	\item Uma lista de tarefas;
	\item As estimativas de tarefas; e,
	\item Atribuições que começarão a equipe no trabalho de desenvolvimento da funcionalidade. A lista de tarefas pode não ser completa, mas deve ser completo o suficiente para refletir o compromisso mútuo por parte de todos os membros da equipe e para carregá-los através da primeira parte do Sprint, enquanto a equipe inventa mais tarefas do Sprint Backlog.
\end{itemize}
Reunião Diária
Há três propósitos para a reunião diária (KEITH, 2010):
\begin{itemize}
	\item Sincronizar esforços entre os membros da equipe;
	\item Se comprometer com o trabalho a ser realizado no dia seguinte, e reafirmar o compromisso da equipe com o objetivo da Sprint;
	\item Garantir que os membros da equipe estão “na mesma página”, ou seja, o time completo precisa ouvir sobre os problemas enfrentados por cada membro para que as soluções possam ser tratadas após a reunião. Permitindo então uma micro orientação do seu progresso em direção ao objetivo em comum.
\end{itemize}

A reunião diária tem tempo fixo de 15 minutos onde todos os membros devem participar, todos devem estar em pé. Como a equipe se reúne em círculo, cada membro deve responder:
1. O que eu fiz desde a reunião diária anterior? (KEITH, 2010)
2. O que eu farei até a próxima reunião diária? (KEITH, 2010)
3. O que me impede de desempenhar meu trabalho eficientemente? (SCHWABER, 2004)
É trabalho do SCRUMMaster garantir que a reunião não seja demorada e ineficaz. Durante a reunião, apenas uma pessoa fala por vez. As demais apenas escutam. Não há conversas paralelas. (SCHWABER, 2004) A reunião diária não é para resolver problemas. Resolver problemas é parte do que ocorre durando todo o dia e é responsabilidade do SCRUMMaster, junto com o time, resolve-los o mais rápido possível. (KEITH, 2010) (MOUNTAINGOAT, 2013)
Reuniões Diárias melhoram as comunicações, eliminam outras reuniões, identificam e removem impedimentos para o desenvolvimento, destacam e promovem rápidas tomadas de decisão, e melhoram o nível de conhecimento do time. Esta é uma reunião chave para inspeção e adaptação. (SCHWABER & SYTHERLAND, 2011)
Revisão da Sprint
A Revisão da Sprint tem tempo fixo de quatro horas. O objetivo é a equipe apresentar ao Product Owner a funcionalidade que está pronta. Embora o significado de "pronto" pode variar de organização para organização, isso normalmente significa que a funcionalidade foi totalmente projetada e poderia potencialmente ser entregue. Funcionalidade que não está "pronta" não pode ser apresentada. (SCHWABER, 2004)
A revisão da Sprint inicia-se com um membro da equipe apresentando o objetivo da Sprint, o Backlog do Produto em que se comprometeram, e o Backlog do Produto concluído. Diferentes membros da equipe podem, então, discutir o que correu bem e o que não correu bem na Sprint. A maior parte do tempo é gasto com os membros da equipe apresentando funcionalidade, respondendo a perguntas dos stakeholders sobre a apresentação, e observando as mudanças que são desejadas. No final das apresentações, os stakeholders ​​são entrevistados, um por um, para obter suas impressões, algumas mudança desejadas, e a prioridade dessas mudanças. (SCHWABER, 2004)
Retrospectiva da Sprint
O tempo fixo da Retrospectiva da Sprint é de três horas sendo de presença exclusiva do time, SCRUMMaster, e do Product Owner, este último opcional. A reunião é iniciada com os membros respondendo a duas perguntas: (SCHWABER, 2004)
1. O que correu bem durante a última Sprint?
2. O que poderia ser melhorado no próxima Sprint?
A Retrospectiva da Sprint é uma oportunidade para o Time inspecionar a si próprio e criar um plano para melhorias a serem aplicadas na próxima Sprint. A Retrospectiva da Sprint ocorre depois da Revisão da Sprint e antes da reunião de planejamento da próxima Sprint. O SCRUM Master motiva o time a melhorar, dentro do processo de desenvolvimento e as práticas para fazê-lo mais efetivo e agradável para a próxima Sprint. (SCHWABER & SYTHERLAND, 2011)
Durante cada Retrospectiva da Sprint, o time planeja formas de aumentar a qualidade do produto, adaptando a definição de “Pronto” quando apropriado. Ao final da Retrospectiva da Sprint, o time deverá ter identificado melhorias que serão implementadas na próxima Sprint. A implementação destas melhorias na próxima Sprint é a forma de adaptação à inspeção que um time faz a si próprio. (SCHWABER & SYTHERLAND, 2011) 
Artefatos do SCRUM
Os artefatos definidos para o SCRUM são especificamente projetados para maximizar a transparência das informações necessárias para assegurar que o time tenha sucesso na entrega do incremento “Pronto”. (SCHWABER & SYTHERLAND, 2011)
Backlog do Produto
É uma lista ordenada e priorizada, pelo Product Owner, de requisitos do projeto com estimativas de esforço para transforma-los em uma funcionalidade completa do produto. A lista evolui a partir de mudanças como de negócio, ou de tecnologia. (SCHWABER, 2004, p. 142)
Os itens primeiramente elucidados representam as necessidades iniciais e melhores conhecidas e entendidas, e evolui constantemente para identificar o que o produto necessita para ser mais apropriado, competitivo e útil. Ele lista todas as características, funções, requisitos, melhorias e correções que formam as mudanças que devem ser feitas no produto nas futuras versões. (SCHWABER & SYTHERLAND, Guia do SCRUM, 2011)
Backlog da Sprint
O Backlog da Sprint define o trabalho ou tarefas que o time define para tornar o Backlog do Produto selecionado para a Sprint corrente em um incremento de funcionalidade do produto potencialmente entregável. O Backlog da Sprint é criado na segunda parte da Reunião de Planejamento da Sprint. As tarefas são divididas de modo em que cada uma leve de quatro horas à 16 horas para terminar. O Backlog da Sprint é uma imagem altamente visível, em tempo real, do trabalho que o time planeja efetuar durante a Sprint. (SCHWABER, 2004)

