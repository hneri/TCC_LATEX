\begin{resumo}
O gerenciamento de projeto é de fundamental importância, pois permite ao gerente monitorar o andamento do projeto nos pontos de interesse da organização, do cliente e das pessoas envolvidas. Entre as métricas de software de produto, processo e de projeto, as de projeto são exploradas para que seja possível manter o projeto sob controle em termos de custo, tempo e escopo. A comparação entre as métodos de gerenciamento tradicional e ágil, PMBOK e SCRUM, respectivamente, fez-se necessária para o estudo das técnicas de Valor Agregado em projeto tradicionais e de Valor Agregado em projeto Ágeis. A aplicação da técnica de Valor Agregado Tradicional em projeto ágeis esbarra em pressupostos que vão na contramão dos valores defendidos pelo manifesto ágil. Logo, utilizando-se da técnica de Valor Agregado Ágil para projetos ágeis de desenvolvimento de software é possível acompanhar o desempenho e progresso do projeto através do monitoramento dos custos. Em posse desses dados, é possível utilizar um ambiente de Data Warehouse para automatizar o processo de extração, transformação, carga (ETL) e visualização de custos oriundas da produção do software.

 \vspace{\onelineskip}
    
 \noindent
 \textbf{Palavras-chaves}: AgileEVM. Data Warehouse. Gerenciamento de Projeto. SCRUM.
\end{resumo}
