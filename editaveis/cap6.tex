\chapter*[Conclusão]{Conclusão}
\addcontentsline{toc}{chapter}{Conclusão}
Com o uso da ferramenta Pentaho PDI foi possível coletar as métricas de uma planilha da ferramenta Excel e também de um banco de dados, no caso, o mySQL, de qualquer tabela nele contido. Com isso é possível observar que o nível de intrusão – interferência humana – diminuiu, pois não mais é necessário coletar métrica por métrica nos diversos locais onde estão armazenadas, e nem com as pessoas que participam de um processo de medição.
Logo, o armazenamento das métricas em um repositório centralizado (não representa ainda um ambiente DW) favorece e contribui para que as métricas, quando seja necessário utiliza-las, estejam de fácil acesso e representam um valor fidedigno quando comparado ao tempo, ou seja, são as métrica mais recentes do processo de medição.
A ferramenta iceScrum satisfaz parcialmente a necessidade de um ScrumMaster que utilizar a técnica AgileEVM, apesar de algumas métricas estarem presentes na ferramenta e serem coletadas e exibidas pelo iceScrum, a necessidade de realização de engenharia reversa foi relevante, pois, juntamente com a ferramenta pentaho, foi possível extrair as informações do banco de dados do iceScrum e portanto calcular as métricas da técnica Agile EVM.
A descrição da fundamentação teórica adquirida durante a realização do projeto foi bastante útil, pois através dela os conceitos de gerenciamento de projeto, tanto tradicional quanto Ágil foram muito importantes e puderam ser aplicados no exemplo de uso da solução através da ferramenta iceScrum.
A diferenciação entre as técnicas de Valor Agregado Tradicional e Ágil contribuíram para justificar e pautar a continuidade do projeto, além de ser possível tornar as métricas mais palpáveis para pessoas que desconhecem o projeto.
Dentro da proposta inicial, o projeto avançou até a fase de ETL, ficando a parte do repositório DW e a visualização para o TCC 2.
No TCC 1 foi realizada a fundamentação teórica, a criação de um plano de métricas – quadros da seção 3.2 -, a familiarização com as ferramentas icescrum e pentaho e a execução de um caso de uso da solução.
Para o TCC2 é necessário aprofundar a fundamentação teórica relacionada a Data Warehouse e a utilização da ferramenta pentaho, com a utilização de ferramentas mondrian e CDE, para que seja possível a implementação de um estudo de caso para validar a proposta do projeto realizando a coleta das informações úteis nos projetos da disciplina GPP/MDS 2013.2.

